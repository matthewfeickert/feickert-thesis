No one does a Ph.D. alone.
I am privileged and lucky to have had the support, encouragement, and guidance by many people throughout my Ph.D., for which I am extremely grateful.
I hope to one day thank everyone for their role in helping me reach this point in my career, but I would like to start here by recognizing a few people.

First and foremost, I want to thank my advisor, Steve Sekula.
I came to SMU to work with Steve and I found the most rewarding professional relationship I could have ever hoped for.
Steve pushed me to pursue work that I was passionate about and went above and beyond to give me opportunities to present my work.
In grad school one if the most important things you can do is choose a good advisor.
I had the best.
I count myself lucky to have had an amazing mentor who was also my friend and look forward to many future collaborations with him.


Thank you to the other members of my Ph.D. thesis committee: Jingbo Ye, Roberto Vega, and Caterina Vernieri.
Your engagement with my research and guidance significantly improved my scholarship as well as the quality of my thesis.

Thank you to the SMU ATLAS group.
Through your support and critique I've become a stronger researcher and I'm proud to have worked with all of you.
Special thanks to Ryszard Stroynowski for always having questions that caused me to engage with the physics deeper and for career mentorship and guidance over the last five years.
I look forward to future physics discussions and bets over wine.

Thank you to Francesco Lo Sterzo and Rafael Teixeira de Lima, the postdocs who mentored me during my time at CERN.
Francesco taught me how to fully engage with physics and did his best to make me a better person as well --- I'd like to think he succeeded.
Rafael taught me how to do a physics analysis, guided me through many tough research questions, was unendingly generous with his time, and a brilliant mentor.
Thank you for making me a better physicist.

Additionally, I need to thank the rest of the SLAC ATLAS group for adopting me for a year.
Michael Kagan welcomingly supported and mentored me and would make time that didn't exist in his schedule to come discuss physics in my office and went out of his way to make me feel included in his group's activities.
Having Michael and Rafael's guidance through my first analysis on ATLAS was invaluable and I’m privileged to have colleagues like them.
I also want to thank Nicole Hartman for always making time to enthusiastically discuss physics, statistics, and machine learning with me over coffee in R1 and for being a font of the most excellent questions.

Thank you to my analysis team for devotedly working with me and helping me do good science for two years.
I am proud of the physics we did together.
Special thanks to Francesco Di Bello for many enlightening discussions and for his camaraderie as we both made our Ph.D.s on this analysis.

Thank you to the my fellow graduate students, colleagues, and friends in the SMU physics department.
Together we studied Wigner matrices, quantum field theory, and became researchers in our own right.
Thank you to Eric Godat for being my office mate and discussing interesting ideas with me in a wide array of fields and topics.
Also thanks to Dan Jardin for working with me to improve the SMU Dedman College Ph.D. Thesis \LaTeX{} template and style enforcement backend that I wrote and for being my thesis writing buddy.
It is mentally quite bolstering in the final pressure filled weeks of writing a dissertation to know that there is someone else working at 3:00 to verify changes with me.
Thank you to Lacey Breaux, who in addition to being a good friend has done more to herd cats and defend myself and other graduate students from impending doom and administrative disaster then I will probably ever know.

Thank you to the CERN lunch crew --- my wonderful friends Dan Guest, Danny Antrim, Sam Meehan, Alex Armstrong, Yvonne Ng, Claire Antel, Serhan Mete, Michela Paganini, Jennifer Roloff, Matt LeBlanc, and Savannah Thais.
I can safely say I learned more physics in lunch conversations with you then at any other time in my life.
Thank you also to Achintya Rao for always making time to sit down with me for passionate discussions over our shared love of open source software and open science.
Thank you to Claire for patiently discussing dark matter physics with me.
Special thanks to Dan, Danny, and Sam for their encouragement to build and experiment and always keeping me honest.

Thank you to Lukas Heinrich and Giordon Stark.
You exemplify what collaborators should be and I'm very proud to have built pyhf with you.
Here's to continuing to build great projects together in the future.
I am immensely honored to call you colleagues, coauthors, and friends.

Thank you to Kyle Cranmer and Mark Neubauer for their support and inclusion of me in the DIANA/HEP and IRIS-HEP projects.
They gave me opportunities to work on things I love and introduced me to an amazing community of likeminded people to share ideas and build software with for the benefit of all of high energy physics.

Finally, thank you to my family.
Thank you to my parents, Dee and Carl, who were the first scientists in my life.
Thank you to Lauren, my wife, my love, my best friend, and my biggest supporter.
Though my path has been nonlinear all of your support has been unwavering.
I would have accomplished none of this without you.
I am forever grateful for what you have given me: everything.
