\chapter{Conclusions}\label{chapter:conclusions}

A search in the high momentum regime for new low mass resonances, produced in association with a jet, decaying into a pair of bottom quarks is presented with a focus on my direct contributions to the analysis.
A short discussion of the results of the analysis and their implications follows.

A search for boosted $\Hbb{} +\mathrm{jet}$ was performed using an integrated $80.5~\ifb$ of proton-proton collisions recorded at ATLAS with a center-of-mass energy of $\sqrt{s} = 13~\TeV$.
Given this data, a measurement of the signal strength of the SM Higgs decaying to $b\bar{b}$ of ${\mu_{H} = 5.8 \pm 3.1~\mathrm{(stat.)} \pm 1.9~\mathrm{(syst.)} \pm 1.7~\mathrm{(th.)}}$ was able to be extracted, corresponding to an measurement that, given uncertainties, is consistent with a background-only hypothesis at $1.6$ standard deviations given the expected sensitivity of $0.28\,\sigma$.
CMS performed a similar analysis in 2017~\cite{CMS:2017cbv} with $35.9~\ifb$ of data and observed a signal strength for $H\to b\bar{b}$ of ${\mu_{H} = 2.3_{-1.6}^{+1.8}}$, which is consistent with this analysis' observation within $2$ standard deviations.
This is the first time this analysis has been performed in ATLAS, and so it is an important advancement of using boosted jet techniques and exploring the usage of new techniques such as variable radius jets.

As this analysis was novel in ATLAS, it has not been fully optimized.
Given the major systematic uncertainties in \Cref{sec:systematic_uncertainties}, it is clear that improvements to the jet mass resolution would significantly improve the analysis.
As the ATLAS calorimeter system is designed to give excellent energy resolution over mass resolution, it will be interesting to see how improvements in jet technologies can improve the analysis.
There is active work in ATLAS to build particle flow into \largeR{} jets, which with the addition of tracks from the ID pointing into the calorimeter would improve the mass resolution of the analysis.
Additionally, the use of new substructure based triggers can improve the signal event selection.

In both the ATLAS and CMS searches of this analysis, a signal strength greater than the Standard Model expectation for the Higgs boson at $m_{H} = 125~\GeV$ has been observed.
Neither of these excesses is significant, given their uncertainties.
However, looking towards the future as both experiments take more data, if the observed signal strength holds at the luminosity weighted mean,
\[
 \braket{\mu} = \sum_{i} f_{i}\,\mu_{i} \pm \left(\sum_{i} \left(f_{i}\sigma_{i}\right)^{2}\right)^{1/2} = 4.7_{-2.82}^{+2.83}, \qquad f_{i} = \frac{L_{i}}{\displaystyle\sum_{i}L_{i}},~i \in \left\{\textrm{ATLAS}, \textrm{CMS}\right\}
\]
then the significance of the observed deviation from the Standard Model expected value of $\mu_{H, \textrm{SM}}=1$ would grow given the proportional decrease%
\footnote{Scaling the statistical uncertainty component of the total uncertainty on the signal strength with increasing luminosity by $\left(L/L_{0}\right)^{-1/2}$.}
in the statistical uncertainty, as shown in \Cref{table:signal_significance_lumi_scaling}.
\Cref{table:signal_significance_lumi_scaling} assumes no improvements to the systematic or theoretical uncertainties, and highlights the integrated luminosities at $\sqrt{s}=13~\TeV$ that the ATLAS analysis will have available at the end of Run 2 of the LHC, the end of Run 3, and the $\sqrt{s}=13~\TeV$ equivalent luminosity%
\footnote{Given a roughly $10\%$ increase in the Higgs production cross section at $\sqrt{s}=14~\TeV$.}
at $\sqrt{s}=14~\TeV$.
It is seen that even with great increases in luminosity the analysis will be limited by the systematic and theoretical uncertainties.
This further motivates the importance of optimizing the analysis and exploring new techniques, in addition to closely watching the improvements of the theoretical community.

\begin{table}[htbp]
 \centering
 \caption[Values of the observed signal significance and uncertainty on the observed signal strength for increasing integrated luminosity.]{%
  Values of the observed signal significance and uncertainty on the observed signal strength for increasing integrated luminosity.
  Both the statistical uncertainty and the total uncertainty are shown.
  The signal significance is given for both the total uncertainty and for the case of only statistical uncertainty.
  Improvements are assumed for only the statistical uncertainty component of the total uncertainty of the analysis.}
 \label{table:signal_significance_lumi_scaling}
 \begin{adjustbox}{max width=\textwidth}
  \begin{tabular}{@{}rrrrrl@{}} \toprule
   $L~\left(\ifb\right)$        & Stat. Uncertainty & Total Uncertainty  & $\mu$ Significance & Stat. Only Sig. & Note                     \\ \midrule
   $140$ at $\sqrt{s}=13~\TeV$  & $\pm1.41$         & $_{-2.26}^{+2.19}$ & $1.64\,\sigma$     & $2.63\,\sigma$  & Full Run 2 dataset       \\
   $440$ at $\sqrt{s}=13~\TeV$  & $\pm0.80$         & $_{-1.94}^{+1.96}$ & $1.92\,\sigma$     & $4.66\,\sigma$  & $300~\ifb$ in Run 3      \\
   $3000$ at $\sqrt{s}=14~\TeV$ & $\pm0.29$         & $_{-1.79}^{+1.81}$ & $2.08\,\sigma$     & $12.77\,\sigma$ & $3300~\ifb$ at $13~\TeV$ \\
   \bottomrule
  \end{tabular}
 \end{adjustbox}
\end{table}

In addition to the measurement of boosted $\Hbb$, a search for low mass leptophobic dark matter mediator $\Zprime$ with democratic vector-axial couplings to the Standard Model quarks was performed using the same dataset.
No significant excess of events is observed in the data, resulting in competitive limits being set for exotic dark matter mediator $\Zprime$ models that exclude at the $95\%$ credibility level mediator models with $g_{q} = 0.25$ below masses of $m_{\Zprime} < 200~\GeV$.
This analysis is not the first form of di-jet analysis in ATLAS to set limits on low mass $\Zprime$, however it sets the most restrictive limits for the low mass search range of $100~\GeV < m_{\Zprime} <170~\GeV$, where~\cite{EXOT-2017-01} has better limits above $170~\GeV$ as that analysis does not have $t\bar{t}$ as a major background.
Given these results, this thesis analysis is an important contribution to the exotic physics jets \& dark matter program in ATLAS and helps to give a comprehensive view of dark matter mediator limits in Run 2 of the LHC.

This thesis analysis has been a successful step forward in bringing burgeoning techniques and new ideas to bear in exploring the wealth of data ATLAS collected in Run 2.
Equipped with this analysis as a tool for inference of Nature, it is with great excitement that I join the particle physics community in preparing for boosted searches of new physics in the upcoming Run 3 of the LHC.
