\chapter{Introduction}\label{chapter:introduction}

The discovery of a Higgs-like boson~\cite{Higgs:1964ia,Higgs:1964pj,Higgs:1966ev,Englert:1964et,Guralnik:1964eu} at CERN in 2012 by the \Gls{ATLAS} and CMS collaborations~\cite{Aad:2012tfa,Chatrchyan:2012xdj} was a major triumph for both theoretical and experimental particle physics.
However, the properties of the new particle remain to be fully verified and the agreement of the predictions of the physics of the Higgs field by the \Gls{Standard Model} (SM) with observations of Nature require further testing.
One important property is the coupling strength of the Higgs boson to bottom quarks $\left(\Hbb\,\right)$ --- this interaction was only experimentally observed in 2018 through associated production with a vector boson~\cite{Aaboud:2018zhk,CMS:2018abb}.
Additionally, there exist models of particle dark matter~\cite{Abdallah:2015ter} which include massive mediators between dark matter and Standard Model particles.
Such \glspl{dark matter mediator} with couplings to Standard Model quarks would have the same decay signature to pairs of bottom quarks as the Higgs.
This thesis presents a search for high-momentum, low-mass resonances, including the Higgs, in the mass range of $100~\GeV$ to $200~\GeV$ decaying to pairs of $b$-quarks with an associated jet $\left(\jXbb\,\right)$.
The goals of the search are to make a direct measurement of the couplings of Higgs to bottom quarks and to search for evidence of exotic resonances with couplings to Standard Model quarks.
The thesis proceeds in the following manner.

\Cref{chapter:theory} introduces the field theories of the Standard Model, describes the physics of the Higgs field, and motivates the search for couplings of the Higgs to $b$-quarks and the search for exotic low-mass resonances.
\Cref{chapter:LHC} introduces CERN's Large Hadron Collider and \Cref{chapter:ATLAS} describes the ATLAS experiment.
\Cref{chapter:event_reconstruction} gives an overview of the techniques used by the ATLAS collaboration to reconstruct the signature of particles in the ATLAS detector for good quality data.
\Cref{chapter:analysis} is devoted to the application of the previous chapters in a search for high-momentum, low-mass resonances with a $b\bar{b}$ final state.
The results of this analysis are presented in \Cref{chapter:results}.
Finally, \Cref{chapter:conclusions} provides a summary of the state of measurements of Higgs couplings to heavy flavor quarks and the search for low-mass exotic resonances given the results of the search, as well as an outlook to physics in Run 3 of the LHC.

My major contributions to the presented analysis are focused in the modeling of the irreducible multijet events from QCD processes --- the dominant background of the analysis.
I developed parametric functions that were able to robustly model the data across all analysis selection regions.
I performed tests to rigorously stress the model stability in the presence of injected signals and its performance with spurious signals.
I additionally quantified the biases associated with each model, informing the choice of additional modeling systematic uncertainties.
