\chapter*{Preface}\label{chapter:preface}
% c.f. https://tex.stackexchange.com/a/222961/152544
\addcontentsline{toc}{chapter}{\protect\numberline{}Preface}

The following is a summary of useful concepts in high energy particle physics.

\section{Units}\label{section:units}

\subsection{Natural Units}\label{subsection:natural_units}

Natural units
\begin{table}[htpb]
 \centering
 \caption{Common quantities in particle physics given in both natural units and SI units.}
 \begin{tabular}{@{}lll@{}} \toprule
  Quantity         & Natural Units & SI Units                                        \\ \midrule
  Speed            & $1$           & $3.0\times 10^{8}~\textrm{m}/\textrm{s}$        \\
  Angular Momentum & $1$           & $10^{34}~\textrm{m}^2 \,\textrm{kg}/\textrm{s}$ \\
  Energy           & \textrm{GeV}  & $1.6\times 10^{-10}~\textrm{J}$                 \\
  Momentum         &               &                                                 \\
  Mass             &               &                                                 \\
  Time             &               &                                                 \\
  Length           &               &                                                 \\
  Electric Charge  &               &                                                 \\
  Magnetic Field   &               &                                                 \\
  \bottomrule
 \end{tabular}\label{table:natural_units}%
\end{table}


\subsection{Units of Luminosity}\label{subsection:luminosity_units}
In high energy physics a crucial quantity of interest is the luminosity (both instantaneous and integrated) delivered by the particle accelerator.
The instantaneous luminosity, $\mathscr{L}$, can be defined as the number of particles incident per unit area per unit time (generally taken to be seconds),
\begin{equation}
 \mathscr{L} = \frac{\text{number of particles}}{\text{unit area} \cdot \text{second}}.
 \label{eq:instantaneous_luminosity}
\end{equation}
In accelerator physics, the unit area is generally chosen to be $\textrm{cm}^2$, giving instantaneous luminosities units of $\textrm{cm}^{-2} \cdot \textrm{s}^{-1}$,
\[
 \left[\mathscr{L}\right] = \textrm{cm}^{-2} \cdot \textrm{s}^{-1}.
\]
However, experimental particle physicists prefer to use units of inverse barns for luminosities. A barn is defined as
\begin{equation}
 1~\textrm{barn} = 10^{-28}~\textrm{m}^2 = 10^{-24}~\textrm{cm}^2,
 \label{eq:barn_to_area}
\end{equation}
which is roughly the cross sectional area of an atomic nuclei.~\cite{web:history_physics_purdue,history:etymology_barn}\\

The context in which luminosity appears as being useful is in the form of an equation like
\[
 N_{\textrm{events}} = \sigma_{\textrm{process}} \cdot \int \mathscr{L}\,dt
\]
where:
\begin{itemize}
 \item $N_{\text{events}}$ is the number of events of a particular process that will be produced at the collider.
       This is what ends up hitting the detector.
 \item $\sigma_{\textrm{process}}$ is the cross section of the particular process to occur per interaction of colliding particles.
       At the LHC this would be the cross section per proton-proton interaction.
       This is a function of the fundamental physics that is available at the energy ranges being probed, and so is also a function of the collider's center of mass energy, $\sqrt{s}$.
       When written in an equation as such it is assumed that the cross section is also including the relevant branching ratios for the final state particles.
       That is, $\sigma\left(H \to b\bar{b}\,\right) = \sigma\left(pp \to H\right) \cdot \mathcal{BR}\left(H \to b\bar{b}\,\right)$.
 \item $L \equiv \int \mathscr{L}\,dt$ is the total luminosity integrated over time (the integrated luminosity).~\cite{Herr:941318}
       This is what is delivered by the particle accelerator.
\end{itemize}

\begin{table}[htpb]
 \centering
 \caption{Instantaneous luminosities at the LHC}
 \begin{tabular}{@{}ll@{}} \toprule
  Description                                                    & Units of $\textrm{cm}^{-2}\cdot\textrm{s}^{-1}$ \\ \midrule
  LHC Design Luminosity~\cite{Bruning:782076}                    & $10^{34}$                                       \\
  2017 ATLAS Trigger Menu Maximum~\cite{TWiki:MenuEvolution2017} & $2.0 \times 10^{34}$                            \\
  2017 Plan~\cite{Indico:MenuCoordination_2017Lumi}              & $1.7 \times 10^{34}$                            \\
  2017 Peak~\cite{TWiki:2017ATLASPeakLumi}                       & $1.7 \times 10^{34}$                            \\
  \bottomrule
 \end{tabular}\label{table:LHC_Luminosity_Goals}%
\end{table}

% Note to self: spacing in tables seems too large. See if can adjust with geometry package
\begin{table}[htpb]
 \centering
 \caption{Luminosity measurements in $\text{barns}^{-1}$}
 % \resizebox{\linewidth}{!}{%
 % \resizebox*{!}{\dimexpr\textheight-2\baselineskip\relax}{%
 \begin{tabular}{@{}lll@{}} \toprule
  Units                & Units of $\textrm{cm}^{-2}$ & Units of $\textrm{fb}^{-1}$ \\ \midrule
  $\textrm{barn}^{-1}$ & $10^{24}$                   & $10^{-15}$                  \\
  $\textrm{mb}^{-1}$   & $10^{27}$                   & $10^{-12}$                  \\
  $\mu\textrm{b}^{-1}$ & $10^{30}$                   & $10^{-9}$                   \\
  $\textrm{nb}^{-1}$   & $10^{33}$                   & $10^{-6}$                   \\
  $\textrm{pb}^{-1}$   & $10^{36}$                   & $10^{-3}$                   \\
  $\textrm{fb}^{-1}$   & $10^{39}$                   & $10^{0}$                    \\
  $\textrm{ab}^{-1}$   & $10^{42}$                   & $10^{3}$                    \\
  \bottomrule
 \end{tabular}\label{table:Luminosity}%
 % }
\end{table}

\section{Coordinates}\label{section:coordinates}

LHC coordinate systems

\section{Statistics}\label{section:statistics}

Statistics in particle physics
